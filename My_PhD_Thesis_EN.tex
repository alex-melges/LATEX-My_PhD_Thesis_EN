\documentclass[12pt,oneside]{book}
\usepackage{graphicx}
\usepackage{subcaption}
\usepackage{color}
\usepackage{amscd,amsfonts,amssymb,amsmath,amsthm}
\usepackage[utf8]{inputenc}
\usepackage[T1]{fontenc}
\usepackage[english]{babel}
\usepackage[all]{xy}
\usepackage{enumerate}
\usepackage[a4paper]{geometry}
\usepackage{fancyhdr}
\usepackage{tabularx}
\usepackage{makeidx}
\usepackage[backend=biber,style=numeric,sorting=nyt]{biblatex}

\addbibresource{referencias.bib}

\geometry{textwidth=150mm,textheight=247mm,top=30mm,bottom=30mm,left=30mm,right=30mm,headsep=10mm,
footskip=10mm}


\pagestyle{fancy}
\fancyhead[LO]{\rightmark}
\fancyhead[RO]{\thepage}
\fancyfoot[C]{\empty}

\makeindex

\newtheorem{defi}   {Definition}[chapter]
\newtheorem{teo}    {Theorem}[chapter]
\newtheorem{ex}     {Example}[chapter]
\newtheorem{prop}   {Proposition}[chapter]
\newtheorem{lem}    {Lemma}[chapter]
\newtheorem{cor}    {Corollary}[chapter]
\newtheorem{obs}    {Observation}[chapter]
\newtheorem{conj}	{Conjecture}

\newcommand{\ds}{\displaystyle}
\newcommand{\R}{\mathbb{R}}
\newcommand{\RP}{\mathbb{R}P}
\newcommand{\Z}{\mathbb{Z}}
\newcommand{\F}{\mathbb{F}}
\newcommand{\ccup}{\smile}
\newcommand{\ccap}{\frown}
\newcommand{\tensor}{\otimes}
\newcommand{\wt}{\widetilde}





\begin{document}





\fontfamily{ptm}\selectfont

\begin{titlepage}
    \begin{center}
        \textbf{FEDERAL UNIVERSITY OF SÃO CARLOS}
    \end{center}

    \vspace{-0.8cm}

    \begin{center}  
        \fontsize{10}{12}\selectfont CENTER OF EXACT SCIENCES AND TECHNOLOGY
    \end{center}

    \vspace{-0.8cm}

    \begin{center}
        \fontsize{10}{12}\selectfont GRADUATE PROGRAM IN MATHEMATICS
    \end{center}

    \vspace{3cm}

    \begin{center}
        \small{Alex Melges Barbosa}
    \end{center}

    \vspace{3cm}

    \begin{center}
        \Large{\textbf{Characteristic Classes of \\ Topological and Generalized Manifolds}}
    \end{center}

    \vspace{10cm}

    \begin{center}
        São Carlos - SP
    \end{center}

    \vspace{-0.8cm}

    \begin{center}
        2022
    \end{center}
\end{titlepage}



\newpage
\thispagestyle{empty}

\begin{center}
    \small{Alex Melges Barbosa}
\end{center}

\vspace{3cm}

\begin{center}
    \Large{\textbf{Characteristic Classes of \\ Topological and Generalized Manifolds}}
\end{center}

\vspace{2cm}

\begin{flushright}
    \begin{minipage}{0.55\textwidth}
        Thesis submitted to the Graduate Program in Mathematics at the Federal University of São Carlos 
        as part of the requirements for the degree of PhD in Mathematics \\[0.5cm] 
        Advisor: Prof. Dr. Edivaldo Lopes dos Santos
    \end{minipage}
\end{flushright}

\vfill

\begin{center}
    São Carlos - SP
\end{center}

\vspace{-0.8cm}

\begin{center}
    2022
\end{center}



\newpage
\thispagestyle{empty}

\vspace{\fill}

\begin{flushright}
    \begin{minipage}{5cm}
        \begin{flushright}
            \vspace{22cm}\textit{To my parents \\ Antonio Carlos and Edith \\ and my siblings \\ 
            Carol and André.}
        \end{flushright}
    \end{minipage}
\end{flushright}



\newpage
\chapter*{Acknowledgements}
\thispagestyle{empty}
\

First of all, I would like to thank God for the opportunities and doors that have opened throughout my 
11-year academic journey, spanning my undergraduate, master’s, and doctoral degrees. There have been 
many moments—both good and challenging—that have shaped me into the person and professional I am today.

I would also like to express my gratitude to my parents, Antonio Carlos and Edith, and my siblings, 
Carol and André, who have been my unwavering support every day.

A special thanks goes to two individuals who have been my mentors throughout this process: Edivaldo and 
João Peres. They have not only helped shape me into the mathematician I am today but, more importantly, 
have played a key role in shaping the person I’ve become. Thank you both very much.

Last but not least, I want to thank my friends, who have been like a second family to me during this 
long journey. There’s no need for me to mention them by name, as only those who have lifted me up during 
my stumbles and celebrated my successes with me truly understand the depth of my gratitude and affection 
for them.

This work was supported financially by CAPES.



\vspace{2cm}

\begin{center}
    \begin{minipage}{10cm}
        \textit{In memory of my father, Antonio Carlos Barbosa. Unfortunately, you couldn't be 
        physically present on the day of my defense, but you certainly were and will always be present 
        in the hearts of our entire family. Thank you for everything!}
    \end{minipage}
\end{center}



\newpage
\thispagestyle{empty}

\begin{flushright}
    \begin{minipage}{7cm}
        \begin{flushright}
            \vspace{22cm}\textit{Either you have a strategy, or you're part of someone else's strategy.}

            \vspace{0.2cm} Alvin Toffler
        \end{flushright}
    \end{minipage}
\end{flushright}



\chapter*{Abstract}
\thispagestyle{empty}

In this work, we will initially present generalized bundles, a concept developed by Fadell with the 
objective of generalizing vector bundles, Stiefel-Whitney classes and Wu's formula from the context of 
smooth manifolds to topological manifolds. After that, we will use the generalized bundles to obtain 
original results of Thom, Stiefel-Whitney, Wu and Euler classes of topological manifolds, as well as 
present a second proof of Wu's formula for topological manifolds and the topological version of the 
Poincaré-Hopf theorem. Finally, we will use the Poincaré and Poincaré-Lefschetz dualities to more 
comprehensively construct the Stiefel-Whitney classes of generalized manifolds in order to present, 
for the first time in the literature, a proof of the Wu's formula for such manifolds.

\

\noindent Keywords: characteristic classes, generalized bundles, topological manifolds, generalized 
manifolds, Wu's formula.



\tableofcontents
\thispagestyle{empty}

\listoffigures
\thispagestyle{empty}



\chapter*{List of notations}
\thispagestyle{empty}

\begin{enumerate}
	\item Saying that $f:X\to Y$ is a map means the same as $f$ being a continuous function between 
    topological spaces.
    \item $f:X \rightleftarrows Y:g$ denotes two maps when $f:X\to Y$ and $g:Y\to X$, not necessarily 
    inverses of each other.
	\item $1:X\to X$ denotes the identity map on $X$.
	\item $f^{-1}$ denotes the preimage of a map $f$, as well as its inverse map (when it exists).
	\item If $f:X\to Y$ is a map, then $f(\_)$ denotes $f(x)$ for every $x\in X$.
	\item If $H:X\times Y\to Z$ is a map defined on a Cartesian product, then $H(\_,y)$ denotes $H(x,y)$ 
    for every $x\in X$. The same applies for $H(x,\_)$.
    
	\item $p_{i}:X_{1}\times \cdots\times X_{n}\to X_{i}$ denota a projeção no $i-$ésimo fator.
	\item $d:X\to X\times X$ denota a aplicação diagonal\index{aplicação!diagonal} dada por $d(x)=(x,x)$.
	\item Dizer que $U\subset X$ é uma vizinhança aberta de algum subconjunto $A\subset X$ significa o mesmo que $U$ ser um subespaço aberto de $X$ que contem A.
	\item Dizer que $\mathcal{U}$ é uma cobertura aberta de um espaço topológico $B$ significa o mesmo que dizer que $\mathcal{U}=\{ U\subset B \}$, de modo que $U\subset B$ é um subespaço aberto de $B$ para todo $U\in\mathcal{U}$ e $\ds\bigcup_{U\in\mathcal{U}}U=B$.
	\item $X \approx Y$ denota quando dois espaços topológicos são homeomorfos.
	\item $f \sim g$ denota quando duas funções são homotópicas.
	\item $X\sim Y$ denota quando dois espaços topológicos tem o mesmo tipo de homotopia.
	\item $G_{1}\cong G_{2}$ denota quando dois objetos algébricos são, apropriadamente, isomorfos.
	\item $\mathbb{S}^{n-1}=\{ x\in\R^{n} \ : \ \| x \|=1 \}$ 	.
	\item $D^{n}=\{ x\in\R^{n} \ : \ \| x \|\leq1 \}$.
	\item $B^{n}=\{ x\in\R^{n} \ : \ \| x \|<1 \}$.								
	\item $I=[0,1]\subset \R$.														
	\item $X^{I}$ denota o espaço topológico dos caminhos em $X$.
	\item $\Omega(X,x_{0})=\{ \omega\in X^{I} \ : \ \omega(0)=\omega(1)=x_{0} \}$
	\item $H_{k}(X,A;R)$ e $H^{k}(X,A;R)$ denotam os $k-$ésimos $R-$módulos de homologia e cohomologia singular, respectivamente, do par $(X,A)$ com coeficientes em um anel $R$ comutativo e com unidade.
	\item $H_{k}^{c}(X,A;R)$ e $H^{k}_{c}(X,A;R)$ denotam, respectivamente, os $k-$ésimos $R-$módulos de homologia e cohomologia singular com suporte compacto.
	\item $\wt{H}_{k}(X,A;R)$ e $\wt{H}^{k}(X,A;R)$ denotam, respectivamente, os $k-$ésimos $R-$módulos de homologia e cohomologia singular reduzidos.
	\item $\check{H}^{k}(X,A;R)$ denota o $k-$ésimo $R-$módulo de cohomologia de \v{C}ech.
	\item $H^{k}(X,A;R)=(x)$ denota que o $k-$ésimo $R-$módulo de cohomologia do par $(X,A)$ é gerado pelo elemento $x\in H^{k}(X,A;R)$. O mesmo para módulos de homologia.
	\item se $x\in H^{k}(X,A;R)$, então denotamos $|x|=k$. O mesmo para módulos de homologia.
	\item $<,>:H^{k}(X,A;R)\tensor H_{k}(X,A;R)\to R$, que associa $\varphi\tensor\sigma\mapsto <\varphi,\sigma>$, denota o produto de Kronecker.
	\item $\ccap:H_{k}(X,A\cup B;R)\tensor H^{l}(X,A;R)\to H_{k-l}(X,B;R)$, que associa $\sigma\tensor\varphi\mapsto\sigma\ccap\varphi$, denota o produto cap.
	\item $\ccup:H^{k}(X,A;R)\tensor H^{l}(X,B;R)\to H^{k+l}(X,A\cup B;R)$, que associa $\varphi\tensor\psi\mapsto\varphi\ccup\psi$, denota o produto cup.
	\item $\times:H^{k}(X,A;R)\tensor H^{l}(Y,B;R)\to H^{k+l}(X\times Y,(X\times B)\cup(A\times Y);R)$, que associa $\varphi\tensor\psi\mapsto\varphi\times\psi$, denota o produto cross.
	\thispagestyle{empty}
\end{enumerate}



\chapter{Introduction}
\thispagestyle{empty}




    %\printbibliography
    %\printindex

    \newpage
    \addcontentsline{toc}{chapter}{Reference Index}
    \thispagestyle{empty}
    \printindex





\end{document}